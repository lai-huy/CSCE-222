% Comment lines start with %
% LaTeX commands start with \
% This template was provided by Jennifer Welch for CSCE 222-200, Honors, Spring 2015

\documentclass[12pt]{article}  % This is an article with font size 12-point

% Packages add features
\usepackage{times}     % font choice
\usepackage{amsmath}   % American Mathematical Association math formatting
\usepackage{amsthm}    % nice formatting of theorems
\usepackage{amssymb}    % provides some symbols
\usepackage{latexsym}  % provides some more symbols
\usepackage{fullpage}  % uses most of the page (1-inch margins)

\setlength{\parskip}{.1in}  % increase the space between paragraphs

\renewcommand{\baselinestretch}{1.1}  % increase the space between lines

% Convenient renaming of symbols for logic formulas
\newcommand{\NOT}{\neg}
\newcommand{\AND}{\wedge}
\newcommand{\OR}{\vee}
\newcommand{\XOR}{\oplus}
\newcommand{\IMPLIES}{\rightarrow}
\newcommand{\IFF}{\leftrightarrow}
\providecommand{\myceil}[1]{$\left \lceil #1 \right \rceil$}
\providecommand{\myfloor}[1]{$\left \lfloor #1 \right \rfloor$}
\newcommand{\powerset}[1]{\mathbb{P}(#1)}

% Actual content starts here.
\begin{document}

\begin{center}         % center all the material between begin and end
{\large                % use larger font
CSCE 222 (Carlisle), Discrete Structures for Computing \\  % \\ is line break
Spring 2022 \\
Homework 10}
\end{center}
\rule{6in}{.1pt}       % horizontal line 6 inches long and .1 point high
\begin{center}
{\large
Type your name below the pledge to sign\\
On my honor, as an Aggie, I have neither given nor received unauthorized aid on this academic work.\\
**HUY QUANG LAI**}
\end{center}

% blank line separates paragraphs.  First line of a paragraph is automatically
% indented.  

\rule{6in}{.1pt}       % horizontal line 6 inches long and .1 point high
                    
\noindent              % don't indent
{\bf Instructions:}    % \bf makes text boldface
                       % \em makes text emphasized (italics)

\begin{itemize}        % makes an itemized list
\item The exercises are from the textbook.  You are encouraged to work
      extra problems to aid in your learning; remember, the solutions to 
      the odd-numbered problems are in the back of the book.
\item Grading will be based on correctness, clarity, and whether your
      solution is of the appropriate length.
\item Always justify your answers.
\item Don't forget to acknowledge all sources of assistance in the section below, and write up your solutions on your own.
\item {\em Turn in .pdf file to Gradescope by the start of class on Monday, April 4, 2022.}  It is simpler to put each problem on its own page using the LaTeX clearpage command.
\end{itemize}


\rule{6in}{.1pt}       % horizontal line 6 inches long and .1 point high

{\bf Help Received:}    % \bf makes text boldface
\begin{itemize}
\item Rosen, Kenneth H. \textit{Discrete Mathematics and Its Applications}. McGraw-Hill, 2019.
\end{itemize}



\rule{6in}{.1pt}       % horizontal line 6 inches long and .1 point high

%---------------------------------------------------------------------

% \subsection makes a subsection heading; * leaves it unnumbered.
% (Usually subsections are inside sections, but the \section command
% used a font that was larger than I wanted.)
\subsection*{Exercises for Section 8.1:}     

\noindent
{\bf 8(a-c): (2 points).}

\noindent
Find a recurrence relation for the number of bit strings of length $n$ that contain three consecutive 0s.

\noindent
Consider a string of length $n\geq3$ that contains three consecutive
0s. Such a string either ends with $1$, or with $10$, or with $100$, or
with $000$. In the first case, there are $s_{n-1}$ possibilities. In the
second case, there are $s_{n-2}$ possibilities. In the third case, there
are $s_{n-3}$ possibilities. And, in the fourth case, there are $2^{n-3}$
possibilities. Hence the recurrence relation is
\[s_n=s_{n-1}+s_{n-2}+s_{n-3}+2^{n-3},n\geq3\]
\line(1,0){450}

\noindent
What are the initial conditions?\\
$s_0=s_1=s_2=0$\\
\line(1,0){450}

\noindent
How many bit strings of length seven contain three consecutive 0s?\\
$s_3=s_2+s_1+s_0+2^0=1$\\
$s_4=s_3+s_2+s_1+2^1=3$\\
$s_5=s_4+s_3+s_1+2^2=8$\\
$s_6=s_5+s_4+s_3+2^3=20$\\
$s_7=s_6+s_5+s_4+2^4=47$

\clearpage
\noindent
{\bf 20(a-b): (2 points).}

\noindent
A bus driver pays all tolls, using only nickels and dimes, by throwing one coin at a time into the mechanical toll collector.

\noindent
Find a recurrence relation for the number of different ways the bus driver can pay a toll of $n$ cents (where the order in which the coins are used matters).

\noindent
Let $P_n$ be the number of ways the bus driver can pay a toll of $n$ cents by using nickels and dimes (where the order in which the coins are used matters).\\
If the toll can be divided by $5$, i.e. $n=5k$ where $k\in\mathbb{N}$, then the number of ways the bus driver can pay the toll is $P_{5k}$.

\noindent
If the toll cannot be divided by $5$, i.e. $n=5k+1,5k+2,5k+3,$ or $5k+4$ where $k\in\mathbb{N}$, then the number of ways the bus driver can pay the toll is $P_{5(k+1)}$

\noindent
Let $P_m=P_{5k}$\\
Then $P_{5(k-1)}=P_{m-1}$ and $P_{5(k-2)}=P_{m-2}$\\
Therefore, the recurrence relation is $P_m=P_{m-1}+P_{m-2}$, where $P_0=1,P_1=1$\\
\line(1,0){450}

\noindent
In how many different ways can the driver pay a toll of 45 cents?

\noindent
$m=9$\\
$P_0=1$\\
$P_1=1$\\
$P_2=P_0+P_1=2$\\
$P_3=P_1+P_2=3$\\
$P_4=P_2+P_3=5$\\
$P_5=P_3+P_4=8$\\
$P_6=P_4+P_5=13$\\
$P_7=P_5+P_6=21$\\
$P_8=P_6+P_7=34$\\
$P_9=P_7+P_8=55$

\clearpage
\noindent
{\bf 56(b-e): (3 points).}

\noindent
In this exercise we will develop a dynamic programming algorithm for finding the maximum sum of consecutive terms of a sequence of real numbers. That is, given a sequence of real numbers $a_1,a_2,\cdots, a_n$, the algorithm computes the maximum sum $\displaystyle\sum_{i=j}^k a_i$ where $1\leq j\leq k\leq n$.

\noindent
Let $M(k)$ be the maximum of the sums of consecutive terms of the sequence ending at $a_k$. That is, $\displaystyle M(k)=\textrm{max}_{1\leq j\leq k}\sum_{i=j}^k a_i$. Explain why the recurrence relation $M(k)=\textrm{max}(M(k-1)+a_k,a_k)$ holds for $k=2,\cdots,n$.

\noindent
$M(k)=\textrm{max}(M(k-1)+a_k,a_k)$\\
$M(k)=\textrm{max}(\textrm{max}(M(k-2+a_{k-1},a+{k-1}))+a_k,a_k)$
$k=1$ is not allowed since $k-1=0$ and there is no such term for $M(0)$.
Looking at the recursion, we find that there are all sums $a_k,a_k+a_{k-1},\cdots a_k+\cdot+a_1$ in the set inside max function. Hence, the iteration will give the maximum of sums consecutive terms of the sequence ending at $a_k$, and the recurrence relation holds.\\
\line(1,0){450}

\noindent
Use part (b) to develop a dynamic programming algorithm for solving this problem.
\begin{verbatim}
m=a_1
for i:=2 to n:
    t=a_i
    for j:=i to 2:
        t = t+a_{j+1}
        m = max(m,t,a_j)
return m
\end{verbatim}
\line(1,0){450}

\noindent
Show each step your algorithm from part (c) uses to find the maximum sum of consecutive terms of the sequence $2,-3,4,1,-2,3$.

\noindent
$a_1=2,a_2=-3,a_3=4,a_4=1,a_5=-2,a_6=3$

\noindent
\textbf{Starting with algorithm considering the first and second term:}\\
$m=2,i=2,t=-3$\\
Running $j$-loop one time\\
$j=2,t=2+(-3)=-1$\\
$m=\textrm{max}(2,-3,2)=2$

\noindent
\textbf{Running algorithm with first three terms:}\\
$i=3,t=4$\\
Running $j$-loop two times\\
$j=3,t=4+(-3)=1$\\
$m=\textrm{max}(2,1,-3)$

\noindent
$j=2,t=1+2=3$\\
$m=\textrm{max}(2,3,-3)=3$

\noindent
\textbf{Running algorithm with first four terms:}\\
$i=4,t=1$\\
Running $j$-loop three times\\
$j=4,t=1+4=5$\\
$m=\textrm{max}(3,5,1)=5$

\noindent
$j=3,t=5+(-3)=2$\\
$m=\textrm{max}(5,2,1)=5$

\noindent
$j=2,t=2+2=4$\\
$m=\textrm{max}(5,4,1)=5$

\noindent
\textbf{Running algorithm with first five terms:}\\
$i=5,t=-2$\\
Running $j$-loop four times\\
$j=5,t=-1+4=3$\\
$m=\textrm{max}(5,-1,2)=5$

\noindent
$j=4,t=-1+4=3$\\
$m=\textrm{max}(5,3,-2)=5$

\noindent
$j=3,t=3+(-3)=0$\\
$m=\textrm{max}(5,0,-2)=5$

\noindent
$j=2,t=0+2=2$\\
$m=\textrm{max}(5,2,-2)=5$

\noindent
\textbf{Running algorithm with first six terms:}\\
$i=6,t=3$
Running $j$-loop five times\\
$j=6,t=3+(-2)=1$\\
$m=\textrm{max}(5,1,3)=5$

\noindent
$j=5,t=1+1=2$\\
$m=\textrm{max}(5,2,-2)=5$

\noindent
$j=4,t=2+4=6$\\
$m=\textrm{max}(5,6,-2)=6$

\noindent
$j=3,t=6+(-3)=3$\\
$m=\textrm{max}(6,3,-2)=6$

\noindent
$j=3,t=3+2=5$\\
$m=\textrm{max}(6,5,-2)=6$

\noindent
The maximum sum of consecutive terms is 6.\\
\line(1,0){450}

\noindent
Show that the worst-case complexity in terms of the number of additions and comparisons of your algorithm from part (c) is linear.

\noindent
Starting with $n=2$ there is one addition and one comparison.\\
Next, for $n=3$, there are two additions and two comparisons.\\
Next, for $n=4$, there are three additions and three comparisons.\\
Next, for $n=5$, there are four additions and four comparisons.

\noindent
As $n$ increases by one, the number of additions and comparisons also increases by 1. Because of this relationship, the worse-case complexity of the algorithm is linear.

\clearpage
\subsection*{Exercises for Section 8.2:}     
\noindent
{\bf 4(a,c,e,g): (2 points).}

\noindent
Solve these recurrence relations together with the initial conditions given.

\noindent
$a_n=a_{n-1}+6a_{n-2}$ for $n\geq2,a_0=3,a_1=6$\\
$a_n-a_{n-1}-6_{n-2}=0$\\
$r^2-r-6r=0$\\
$(r-3)(r+2)$\\
$r=\{-2,3\}$\\
$a_n=c(-2)^n+d(3)^n$
$c+d=3,-2c+3b=6$\\
$\displaystyle c=\frac{3}{5},d=\frac{12}{5}$\\
$\displaystyle a_n=\frac{3}{5}(-2)^n+\frac{12}{5}(3)^n$\\
\line(1,0){450}

\noindent
$a_n=6a_{n-1}-8a_{n-2}$ for $n\geq2,a_0=4,a_1=10$\\
$a_n-6a_{n-1}+8a_{n-2}=0$\\
$r^2-6r+8=0$\\
$(r-4)(r-2)=0$\\
$r=\{2,4\}$\\
$a_n=c(2)^n+d(4)^n$\\
$c+d=4,2c+4d=10$\\
$c=3,d=1$\\
$a_n=3(2)^n+1(4)^n$\\
\line(1,0){450}

\noindent
$a_n=a_{n-2}$ for $n\geq2,a_0=5,a_1=-1$\\
$a_n-a_{n-2}=0$\\
$r^2-1=0$\\
$r=\pm1$\\
$a_n=c(-1)^n+d(1)^n$\\
$c+d=5,-c+d=-1$\\
$c=-3,d=7$\\
$a_n=-3(-1)^n+7(1)^n$\\
\line(1,0){450}

\clearpage
\noindent
$a_{n+2}=-4a_{n+1}+5a_n$ for $n\geq0,a_0=2,a_1=8$\\
$a_{n+2}+4a_{n+1}-5a_n=0$\\
$r+4r^2-5=0$\\
$(r-1)(r+5)=0$\\
$r=\{-5,1\}$\\
$a_n=c(-5)^n+d(1)^n$\\
$c+d=2,-5c+d=10$\\
$c=-1,d=3$\\
$a_n=-1(-5)^n+3(1)^n$

\noindent
{\bf 28(a-b): (2 points).}

\noindent
Find all solutions of the recurrence relation $a_n=2a_{n-1}+2n^2$.

\noindent
$a_n-2a_{n-1}=0$\\
$r-2=0,r=2$\\
$a_n=c(2)^n$

\noindent
$a_n=c(2)^n+p_2n^2+p_1n+p_0$
\begin{align*}
a_n             &= 2a_{n-1}+2n^2                    \\
p_2n^2+p_1n+p_0 &= 2(p_2(n-1)^2+p_1(n-1)+p_0)+2n^2  \\
(p_2+2)n^2+(p_1-4p_2)n+(2p_2-2p_1+p_0) &= 0         \\
(p_2+2)n^2+(p_1-4p_2)n+(2p_2-2p_1+p_0) &= 0n^2+0n+0
\end{align*}

\noindent
$p_2+2=0,p_2=-2$\\
$p_1-4p_2=0,p_1=-8$\\
$2p_2-2p_1+p_0=0$\\
$-4+16+p_0,p_0=-12$

\noindent
$a_n=c(2)^n-2n^2-8n-12$\\
\line(1,0){450}

\noindent
Find the solution of the recurrence relation in part (a) with initial condition $a_1=4$.\\
$4=c(2)^1-2(1^2)-8-12$\\
$4=2c-22$\\
$c=13$

\noindent
$a_n=13\cdot2^n-2n^2-8n-12$

\clearpage
\subsection*{Exercises for Section 8.3:}     
\noindent
{\bf 18(a-b): (2 points).}

\noindent
Suppose that each person in a group of $n$ people votes for exactly two people from a slate of candidates to fill two positions on a committee. The top two finishers both win positions as long as each receives more than $\displaystyle\frac{n}{2}$ votes.

\noindent
Devise a divide-and-conquer algorithm that determines whether the two candidates who received the most votes each received at least $\displaystyle\frac{n}{2}$ votes and, if so, determine who these two candidates are.

\noindent
The base case is that a sequence with one element means that the one person on the list is the winner.\\
For the recursive step, divide the list into two equal parts and count which name occurs the most in the two parts.\\
The winner requires a majority of votes and will need at least $\displaystyle\frac{n}{2}+1$ votes.\\
Keep applying this recursive step to each half until the list contains at most two names.\\
Count the number of occurrences in the whole list of the two remaining names and this will decide the winner.\\
\line(1,0){450}

\noindent
Use the master theorem to give a big-$O$ estimate for the number of comparisons needed by the algorithm you devised in part (a).

\noindent
The function would be $\displaystyle f(n)=2f\left(\frac{n}{2}\right)+2n$. So $a=2,b=2,c=2,d=1$.\\
By the master theorem, $a=b^d$ so the big-$O$ is $O(n^d\log{n})=O(n\log{n})$

\clearpage
\noindent
{\bf 22: (2 points).}

\noindent
Suppose that the function $f$ satisfies the recurrence relation $f(n)=2f(\sqrt{n})+\log{n}$ whenever $n$ is a perfect square greater than $1$ and $f(2)=1$.

\noindent
Find $f(16)$.\\
$f(4)=2f(2)+\log{4}=4$\\
$f(16)=2f(4)+\log{16}$\\
$f(16)=12$

\noindent
Find a big-$O$ estimate for $f(n)$.\\
Let $m=\log{n},n=2^m$\\
$\displaystyle f(2^m)=2f(\sqrt{2^m})+\log{(2^m)}$\\
$\displaystyle f(2^m)=2f\left(2^\frac{m}{2}\right)+m$

\noindent
$\displaystyle T(m)=2T\left(\frac{m}{2}\right)+m$\\
$a=2,b=2,d=1$\\
$a=b^d$\\
$O(m^d\log{m})$\\
$O(\log{n}\cdot\log{\left(\log{n}\right)})$

\clearpage
\subsection*{Exercises for Section 8.4:}     
\noindent
{\bf 12(a,c,e): (3 points).}

\noindent
Find the coefficient of $x^{12}$ in the power series of each of these functions.

\noindent
$\displaystyle\frac{1}{1+3x}$
\begin{flalign*}
    \frac{1}{1+3x}  &= \frac{1}{1-(-3x)}            &\\
                    &= \sum_{k=0}^{+\infty}(-3x)^k  &
\end{flalign*}
$(-3)^{12}$\\
\line(1,0){450}

\noindent
$\displaystyle\frac{1}{(1+x)^8}$
\begin{flalign*}
    \frac{1}{(1+x)^8}   &= \frac{1}{(1-(-x))^8} &\\
                        &= \sum_{k=0}^{+\infty}\binom{7+k}{k}(-x)^k
\end{flalign*}
$\displaystyle\binom{19}{12}(-1)^{12}$\\
\line(1,0){450}

\noindent
$\displaystyle\frac{x^3}{(1+4x)^2}$
\begin{flalign*}
    \frac{x^3}{(1+4x)^2}    &= x^3\cdot\frac{1}{(1-(-4x))^2}    &\\
                            &= x^3\cdot\sum_{k=0}^{\infty}\binom{1+k}{k}(-4x)^k &\\
                            &= \sum_{k=0}^{\infty}\binom{1+k}{k}(-4)^kx^{k+3}
\end{flalign*}
$\displaystyle\binom{10}{9}(-4)^9$

\clearpage
\noindent
{\bf 14: (2 points).}

\noindent
Use generating functions to determine the number of different ways 12 identical action figures can be given to five children so that each child receives at most three action figures.

\noindent
$(1+x+x^2+x^3)^5$ coefficient of $x^{12}$\\
There are 35 ways of distributing 12 identical action figures to five children so that each child receives at most three action figures

\end{document}
