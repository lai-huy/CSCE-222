% Comment lines start with %
% LaTeX commands start with \
% This template was provided by Jennifer Welch for CSCE 222-200, Honors, Spring 2015

\documentclass[12pt]{article}  % This is an article with font size 12-point

% Packages add features
\usepackage{times}     % font choice
\usepackage{amsmath}   % American Mathematical Association math formatting
\usepackage{amsthm}    % nice formatting of theorems
\usepackage{amssymb}    % provides some symbols
\usepackage{latexsym}  % provides some more symbols
\usepackage{fullpage}  % uses most of the page (1-inch margins)

\setlength{\parskip}{.1in}  % increase the space between paragraphs

\renewcommand{\baselinestretch}{1.1}  % increase the space between lines

% Convenient renaming of symbols for logic formulas
\newcommand{\NOT}{\neg}
\newcommand{\AND}{\wedge}
\newcommand{\OR}{\vee}
\newcommand{\XOR}{\oplus}
\newcommand{\IMPLIES}{\rightarrow}
\newcommand{\IFF}{\leftrightarrow}
\providecommand{\ceil}[1]{$\left \lceil #1 \right \rceil$}
\providecommand{\floor}[1]{$\left \lfloor #1 \right \rfloor$}
\newcommand{\perm}[2]{{}^{#1}\!P_{#2}}%
\newcommand{\comb}[2]{{}^{#1}C_{#2}}%
\newcommand{\powerset}[1]{\mathbb{P}(#1)}

% Actual content starts here.
\begin{document}

\begin{center}         % center all the material between begin and end
{\large                % use larger font
CSCE 222 (Carlisle), Discrete Structures for Computing \\  % \\ is line break
Spring 2022 \\
Homework 9}
\end{center}
\rule{6in}{.1pt}       % horizontal line 6 inches long and .1 point high
\begin{center}
{\large
Type your name below the pledge to sign\\
On my honor, as an Aggie, I have neither given nor received unauthorized aid on this academic work.\\
**HUY QUANG LAI**}
\end{center}

% blank line separates paragraphs.  First line of a paragraph is automatically
% indented.  

\rule{6in}{.1pt}       % horizontal line 6 inches long and .1 point high
                    
\noindent              % don't indent
{\bf Instructions:}    % \bf makes text boldface
                       % \em makes text emphasized (italics)

\begin{itemize}        % makes an itemized list
\item The exercises are from the textbook.  You are encouraged to work
      extra problems to aid in your learning; remember, the solutions to 
      the odd-numbered problems are in the back of the book.
\item Grading will be based on correctness, clarity, and whether your
      solution is of the appropriate length.
\item Always justify your answers.
\item Don't forget to acknowledge all sources of assistance in the section below, and write up your solutions on your own.
\item {\em Turn in .pdf file to Gradescope by the start of class on Monday, March 28, 2022.}  It is simpler to put each problem on its own page using the LaTeX clearpage command.
\end{itemize}


\rule{6in}{.1pt}       % horizontal line 6 inches long and .1 point high

{\bf Help Received:}    % \bf makes text boldface
\begin{itemize}
\item Rosen, Kenneth H. \textit{Discrete Mathematics and Its Applications}. McGraw-Hill, 2019.
\end{itemize}



\rule{6in}{.1pt}       % horizontal line 6 inches long and .1 point high

%---------------------------------------------------------------------

% \subsection makes a subsection heading; * leaves it unnumbered.
% (Usually subsections are inside sections, but the \section command
% used a font that was larger than I wanted.)
\subsection*{Exercises for Section 6.1:}     

\noindent
{\bf 8: (1 point).}\\
How many different three-letter initials with none of the letters repeated can people have?

\noindent
$\displaystyle26\cdot25\cdot24=15600$

\noindent
{\bf 16: (2 points).}\\
How many strings are there of four lowercase letters that have the letter $x$ in them?\\
There are a total of $26^4$ four-letter strings. There are $25^4$ strings that do not contain an $x$.\\
Therefore, the total number of strings that contain the letter $x$ is $26^4-25^4$ or $66351$.

\noindent
{\bf 28: (1 point).}\\
\noindent
How many license plates can be made using either three digits followed by three uppercase English letters or three uppercase English letters followed by three digits?\\
There are 10 digits and 26 uppercase English letters.\\
Therefore, the total number of license plates that can be made are $2\cdot10^3\cdot26^3$ or $35152000$.

\noindent
{\bf 48c: (1 point).}\\
In how many ways can a photographer at a wedding arrange 6 people in a row from a group of 10 people, where the bride and the groom are among these 10 people, if exactly one of the bride and the groom is in the picture?

\noindent
Since either the bride or the groom be in the picture, that leaves 5 more people to be chosen from the group of 10. Therefore, the total number of wedding arrangements is $\displaystyle2\binom{10}{5}$ or $504$.

\clearpage
\subsection*{Exercises for Section 6.2:}     
\noindent
{\bf 18: (2 points).}\\
How many numbers must be selected from the set $\{1,3,5,7,9,11,13,15\}$ to guarantee that at least one pair of these numbers add up to 16?

\noindent
We can split the list into the following four groups: $\{1,15\},\{3,13\},\{5,11\},\{7,9\}$. Using the pigeon hole principle, we can one number from each pair that will not add up to 16. Therefore, at least 5 numbers must be chosen from the list to guarantee that at least one pair of the numbers adds up to 16.

\noindent
{\bf 20b: (2 points).}\\
\noindent
Suppose that there are nine students in a discrete mathematics class at a small college. Show that the class must have at least three male students or at least seven female students.

\noindent
Assume that the class has less than 3 male and less than seven female students.
Then the number of students in the class is: $S=M+F=2+6=8$.\\
To add an additional student into the class to get a total of nine students, the student has to be either male or female and would cause there to be at least three male students or at least seven female students.


\noindent
{\bf 40: (2 points).}\\
\noindent
Find the least number of cables required to connect eight computers to four printers to guarantee that for every choice of four of the eight computers, these four computers can directly access four different printers. Justify your answer.

\noindent
Connect $C_1$ to $P_1$, $C_2$ to $P_2$, $C_3$ to $P_3$, and $C_4$ to $P_4$. This will create four connections.

\noindent
Next you connect $C_5$ through $C_8$ to the four printers. This will create an additional sixteen connections.

\noindent
The total number of connections is 20. With these connections, it is guaranteed through the pigeon hole principle that any four computers are connected to four different printers.

\clearpage
\subsection*{Exercises for Section 6.3:}     

\noindent
{\bf 30: (1 point).}\\
\noindent
A professor writes 40 discrete mathematics true/false questions. Of the statements in these questions, 17 are true. If the questions can be positioned in any order, how many different answer keys are possible?

\noindent
$\displaystyle\binom{40}{17}$

\noindent
{\bf 32b: (1 point).}\\
Seven women and nine men are on the faculty in the mathematics department at a school. How many ways are there to select a committee of five members of the department if at least one woman and at least one man must be on the committee?

\noindent
{\bf 38: (2 points).}\\
How many bit strings contain exactly five 0s and 14 1s if every 0 must be immediately followed by two 1s?

\noindent
Since every 0 must be immediately followed by two 1s, there exists five blocks of ``011'' and four 1s. This gives us nine objects to order.

\noindent
Since every block of ``011'' can be swapped without changing the string, we can divide out the number of these arrangements. Similarly, we can divide out the number of arrangements of the ''1'' blocks.

\noindent
Therefore, the number of strings that fit the criteria is $\displaystyle\frac{9!}{4!\cdot5!}=\binom{9}{4}=\binom{9}{5}$.

\clearpage
\subsection*{Exercises for Section 6.4:}     

\noindent
{\bf 8: (1 point).}\\
What is the coefficient of $x^8y^9$ in the expansion of $(3x+2y)^{17}$?

\noindent
$\displaystyle(x+y)^n=\sum_{k=0}^n\binom{n}{k}x^{n-k}y^k$\\
$\displaystyle\binom{17}{8}(3x)^8(2y)^9$\\
The coefficient would be $\displaystyle\binom{17}{8}\cdot3^8\cdot2^9$

\noindent
{\bf 26b: (2 points).}\\
Prove the identity $\displaystyle\binom{n}{r}\binom{r}{k}=\binom{n}{k}\binom{n-k}{r-k}$, whenever $n$, $r$, and $k$ are non-negative integers with $r\leq n$ and $k\leq r$, using an argument based on the formula for the number of $r$-combinations of a set with $n$ elements.

\noindent
{\bf 32b: (2 points).}\\
Show that if $n$ is a positive integer, then $\displaystyle\binom{2n}{n}=2\binom{n}{2}+n^2$ by algebraic manipulation.

\noindent
$\displaystyle\binom{2n}{2}=\frac{(2n)!}{2!(2n-2)!}$\\
$\displaystyle=\frac{(2n)(2n-1)}{2}=2n^2-n$

\noindent
$\displaystyle2\binom{n}{2}+n^2=2\frac{n!}{2!(n-2)!}+n^2$\\
$\displaystyle=n(n-1)+n^2=2n^2-n$

\noindent
$\because\displaystyle\binom{2n}{n}=2n^2-n=\binom{n}{2}$\\
$\therefore\displaystyle\binom{2n}{n}=\binom{n}{2}$

\end{document}
