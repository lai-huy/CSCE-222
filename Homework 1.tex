% Comment lines start with %
% LaTeX commands start with \
% This template was provided by Jennifer Welch for CSCE 222-200, Honors, Spring 2015

\documentclass[12pt]{article}  % This is an article with font size 12-point

% Packages add features
\usepackage{times}      % font choice
\usepackage{amsmath}    % American Mathematical Association math formatting
\usepackage{amsthm}     % nice formatting of theorems
\usepackage{amssymb}    % provides some more symbols
\usepackage{latexsym}   % provides some more symbols
\usepackage{fullpage}   % uses most of the page (1-inch margins)

\setlength{\parskip}{.1in}  % increase the space between paragraphs

\renewcommand{\baselinestretch}{1.1}  % increase the space between lines

% Convenient renaming of symbols for logic formulas
\newcommand{\NOT}{\neg}
\newcommand{\AND}{\wedge}
\newcommand{\OR}{\vee}
\newcommand{\XOR}{\oplus}
\newcommand{\IMPLIES}{\rightarrow}
\newcommand{\IFF}{\leftrightarrow}

% Actual content starts here.
\begin{document}

\begin{center}         % center all the material between begin and end
{\large                % use larger font
CSCE 222 (Carlisle), Discrete Structures for Computing \\  % \\ is line break
Spring 2022 \\
Homework 1}
\end{center}
\rule{6in}{.1pt}       % horizontal line 6 inches long and .1 point high
\begin{center}
{\large
Type your name below the pledge to sign\\
On my honor, as an Aggie, I have neither given nor received unauthorized aid on this academic work.\\
HUY QUANG LAI}
\end{center}

% blank line separates paragraphs.  First line of a paragraph is automatically
% indented.  

\rule{6in}{.1pt}       % horizontal line 6 inches long and .1 point high
                    
\noindent              % don't indent
{\bf Instructions:}    % \bf makes text boldface
                       % \em makes text emphasized (italics)

\begin{itemize}        % makes an itemized list
\item The exercises are from the textbook.  You are encouraged to work
      extra problems to aid in your learning; remember, the solutions to 
      the odd-numbered problems are in the back of the book.
\item Grading will be based on correctness, clarity, and whether your
      solution is of the appropriate length.
\item Always justify your answers.
\item Don't forget to acknowledge all sources of assistance in the section below, and write up your solutions on your own.
\item {\em Turn in .pdf file to Gradescope by the start of class on Monday, January 24, 2022.} It is simpler to put each problem on its own page using the LaTeX clearpage command.
\end{itemize}

\rule{6in}{.1pt}       % horizontal line 6 inches long and .1 point high

{\bf Help Received:}    % \bf makes text boldface
\begin{itemize}
\item Discrete Mathematics and Its Applications, 8th Edition
\end{itemize}

\rule{6in}{.1pt}       % horizontal line 6 inches long and .1 point high
\clearpage
\noindent
{\bf LaTeX hints:}  Read this .tex file for some explanations that are in
the comments.

Math formulas are enclosed in \$ signs, e.g., {\tt \$x + y = z\$}
becomes $x + y = z$.

Logical operators: $\NOT, \AND, \OR, \XOR, \IMPLIES, \IFF$.

Here is a truth table using the ``tabular'' environment:

\begin{center}
\begin{tabular}{|c|c|}  % two columns, both centered (c), 
                        % divided by vertical lines (|)
\hline                  % horizontal line
$p$ & $\NOT p$ \\       % separate column entries with &
\hline
\hline
T & F \\
\hline
F & T \\
\hline
\end{tabular}
\end{center}



\rule{6in}{.1pt}       % horizontal line 6 inches long and .1 point high

%---------------------------------------------------------------------

% \subsection makes a subsection heading; * leaves it unnumbered.
% (Usually subsections are inside sections, but the \section command
% used a font that was larger than I wanted.)
\subsection*{Exercises for Section 1.1:}     

\noindent{\bf{8(e): (1 pt)}}
\newline
False.\newline
This statement would imply that Smartphone A has more RAM than itself which is impossible.

\noindent{\bf{12(h): (1 pt)}}
\newline
The votes have not been counted or the election has not been decide and the votes are counted.

\noindent{
    \bf{29(b): (1 pt)}
}                                   \newline
\noindent{
    $p$ = ``There is a quiz."       \newline
    $q$ = ``I will go to class."
}

\begin{align*}
    \textrm{Converse: } & \textrm{If I come to class, then there will be a quiz.} \\
    \textrm{Contrapositive: } & \textrm{If I do not go to class, then there will not be a quiz.} \\
    \textrm{Inverse: } & \textrm{If there is not a quiz, then I will not come to class.}
\end{align*}

\noindent{\bf{34(f): (2 pts)}}

\begin{tabular}{|c|c|c|c|c|}  % two columns, both centered
                        % divided by vertical lines (|)
\hline                  % horizontal line
$p$ & $q$ & $p\IFF q$ & $p\IFF\NOT q$ & 34(f)\\       % separate column entries with &
\hline
F & F & T & F & T\\
\hline
F & T & F & T & T\\
\hline
T & F & F & T & T\\
\hline
T & T & T & F & T\\
\hline
\end{tabular}
\clearpage

%--------------------------------------------------------------------

\subsection*{Exercises for Section 1.2:}     

\noindent{\bf{10: (1 pt)}}
\newline
These system specifications are consistent\newline
The system is not being upgraded.\newline
The user can access the file system.\newline
The user can save new files.\newline

\noindent{\bf{18(c): (1 pt)}}
\newline
The Queen who never lies can state this statement.\newline
Treasure 1 is empty by the statement on Treasure 2.\newline
Treasure 2 is full by the statement on Treasure 3.\newline
Treasure 3 is full by the fact that two out of three treasures are full.

\noindent{\bf 38: (1 pt)}
\begin{enumerate}
    \item $\displaystyle K\OR H$
    \item $\displaystyle R \XOR V$
    \item $\displaystyle A \AND R$
    \item $\displaystyle V \IFF K$
    \item $\displaystyle H \AND (A \AND K)$
\end{enumerate}
You can determine if the 5 members are chatting.
\newline
If Kevin is chatting, then Vijay is chatting by statement 4.
If Vijay is chatting, then Randy is not chatting by statement 2.
If Randy is not chatting, then Abby is not chatting by statement 3.
If Abby is not chatting, then Heather is not chatting by statement 5.

\clearpage
\noindent{\bf 40a: (2 pts)}
\newline
Know: Exactly one person is True
\newline
Statements:
\begin{enumerate}
    \item Alice claims Carlos did it
    \item John claims he did not do it.
    \item Carlos claims Diana did it.
    \item Diana claims that Carlos is false.
\end{enumerate}
John is the hacker.\newline
There's a logical inconsistency for the other options.

If Alice is the one telling the truth, then Carlos is guilty, and he must be lying about Diana. Because of this, Diana is telling the truth. However, only one truth teller can exist so Alice must be lying.

If John is telling the truth, this leaves Alice, Carlos or Diana to be the hacker. If Alice is the hacker, then Diana is telling the truth. If Carlos is the hacker then Alice is telling the truth. If Diana is the hacker, then Carlos is telling the truth. These conclusions, violate the fact that exactly one of them is telling the truth.

If Carlos is telling the truth, then Diana is guilty. In this case John is also telling the truth. Again, exactly one of them can tell the truth.

If Diana is the truth teller, Alice is lying about Carlos' guilt, Carlos is lying about Diana's guilt, and John is lying about his innocence.

\noindent{\bf 44(a): (1 pt)}
\newline
$\displaystyle\NOT{p}\OR{\NOT{q}}$
\clearpage

%--------------------------------------------------------------------

\subsection*{Exercises for Section 1.3:}     

\noindent{\bf{8(c) (1 pt)}}
\newline
James is not young or not strong.

\noindent{\bf{10(c) (2 pts)}}
\newline
$(p\IMPLIES\NOT{q})\IMPLIES(\NOT{p}\IMPLIES q)$\newline
$=\NOT{(\NOT{p}\OR{\NOT{q}})}\OR{(p\OR{q})}$\newline
\newline
Let $P=(p\IMPLIES\NOT{q})$ and ${Q=(\NOT{p}\IMPLIES q)}$\newline
$\NOT{P}\OR{Q}$\newline
$\NOT{(p\IMPLIES{\NOT{q}})}\OR{(\NOT{p}\IMPLIES q)}$\newline
\newline
For $p\IMPLIES{\NOT{q}}$:\newline
$=\NOT{p}\OR{\NOT{q}}$\newline
\newline
For $\NOT{p}\IMPLIES{q}$:
\newline
$=p\OR{q}$\newline

\noindent{\bf{20 (2 pts)}}
\newline
Show that $p\IFF{q}$ and $(p\AND{q})\OR{(\NOT{p}\AND{\NOT{q}})}$ are logically equivalent.


\begin{tabular}{c|c|c}
    $p$ & $q$ & $p\IFF q$  \\
    \hline
    0 & 0 & 1 \\
    \hline
    0 & 1 & 0 \\
    \hline
    1 & 0 & 0 \\
    \hline
    1 & 1 & 1 \\
\end{tabular}

\begin{tabular}{c|c|c|c|c}
    $p$ & $q$ & $p\AND{q}$ & $\NOT{p}\AND\NOT{q}$ & $(p\AND q)\OR(\NOT{p}\AND\NOT q)$ \\
    \hline
    0 & 0 & 0 & 1 & 1\\
    \hline
    0 & 1 & 0 & 0 & 0\\
    \hline
    1 & 0 & 0 & 0 & 0\\
    \hline
    1 & 1 & 1 & 0 & 1\\
\end{tabular}

$\therefore p\IFF q\equiv(p\AND q)\OR(\NOT{p}\AND\NOT q)$
\clearpage

%--------------------------------------------------------------------

\subsection*{Exercises for Section 1.4:}     

\noindent{\bf{10(e): (1 pt)}}
\newline
Let $C(x)$ be the statement “$x$ has a cat,” let $D(x)$ be the
statement “$x$ has a dog,” and let $F(x)$ be the statement “$x$
has a ferret.” Express each of these statements in terms
of $C(x)$, $D(x)$, $F(x)$, quantifiers, and logical connectives.
Let the domain consist of all students in your class.
\newline
For each of the three animals, cats, dogs, and ferrets,
there is a student in your class who has this animal as
a pet.

$(\exists xC(x))\land(\exists xD(x))\land(\exists xF(x))$

\noindent{\bf{42(b): (1 pt)}}
\newline
Express each of these system specifications using predicates, quantifiers, and logical connectives.\newline
No directories in the file system can be opened and no files can be closed when system errors have been detected.\newline
Let $D(x)$ be if the directory in the file system can be opened.\newline
Let $F(x)$ be if file system can be closed.\newline
Let $E$ be if a system error has been detected.\newline

$E\IMPLIES(\forall x\neg{D(x)}\AND{\forall x \neg{F(x)}})$

\noindent{\bf{46: (2 pts)}}
\newline
Determine whether $\forall x (P(x)\IFF{Q(x)})$ and $\forall x P(x)\IFF{\forall xQ(x)}$ are logically equivalent. Justify your answer.\newline
\newline
$P(1)=T, P(2)=F, P(3)=T$\newline
$Q(1)=T, Q(2)=T, Q(3)=T$\newline
Given these conditions, the $\forall x P(x)\IFF{\forall xQ(x)}$ is always false.\newline
However, $\forall x (P(x)\IFF{Q(x)})$ is a conditional proposition.\newline
Given this, $\forall x (P(x)\IFF{Q(x)})\not\equiv\forall x P(x)\IFF{\forall xQ(x)}$

\clearpage

%--------------------------------------------------------------------

\end{document}