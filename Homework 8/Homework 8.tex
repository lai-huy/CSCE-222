% Comment lines start with %
% LaTeX commands start with \
% This template was provided by Jennifer Welch for CSCE 222-200, Honors, Spring 2015

\documentclass[12pt]{article}  % This is an article with font size 12-point

% Packages add features
\usepackage{times}     % font choice
\usepackage{amsmath}   % American Mathematical Association math formatting
\usepackage{amsthm}    % nice formatting of theorems
\usepackage{amssymb}    % provides some symbols
\usepackage{latexsym}  % provides some more symbols
\usepackage{fullpage}  % uses most of the page (1-inch margins)

\setlength{\parskip}{.1in}  % increase the space between paragraphs

\renewcommand{\baselinestretch}{1.1}  % increase the space between lines

% Convenient renaming of symbols for logic formulas
\newcommand{\NOT}{\neg}
\newcommand{\AND}{\wedge}
\newcommand{\OR}{\vee}
\newcommand{\XOR}{\oplus}
\newcommand{\IMPLIES}{\rightarrow}
\newcommand{\IFF}{\leftrightarrow}
\providecommand{\myceil}[1]{$\left \lceil #1 \right \rceil$}
\providecommand{\myfloor}[1]{$\left \lfloor #1 \right \rfloor$}
\newcommand{\powerset}[1]{\mathbb{P}(#1)}

% Actual content starts here.
\begin{document}

\begin{center}         % center all the material between begin and end
{\large                % use larger font
CSCE 222 (Carlisle), Discrete Structures for Computing \\  % \\ is line break
Spring 2022 \\
Homework 8}
\end{center}
\rule{6in}{.1pt}       % horizontal line 6 inches long and .1 point high
\begin{center}
{\large
Type your name below the pledge to sign\\
On my honor, as an Aggie, I have neither given nor received unauthorized aid on this academic work.\\
**HUY QUANG LAI**}
\end{center}

% blank line separates paragraphs.  First line of a paragraph is automatically
% indented.  

\rule{6in}{.1pt}       % horizontal line 6 inches long and .1 point high
                    
\noindent              % don't indent
{\bf Instructions:}    % \bf makes text boldface
                       % \em makes text emphasized (italics)

\begin{itemize}        % makes an itemized list
\item The exercises are from the textbook.  You are encouraged to work
      extra problems to aid in your learning; remember, the solutions to 
      the odd-numbered problems are in the back of the book.
\item Grading will be based on correctness, clarity, and whether your
      solution is of the appropriate length.
\item Always justify your answers.
\item Don't forget to acknowledge all sources of assistance in the section below, and write up your solutions on your own.
\item {\em Turn in .pdf file to Gradescope by the start of class on Monday, March 21, 2022.}  It is simpler to put each problem on its own page using the LaTeX clearpage command.
\end{itemize}


\rule{6in}{.1pt}       % horizontal line 6 inches long and .1 point high

{\bf Help Received:}    % \bf makes text boldface
\begin{itemize}
\item Rosen, Kenneth H. \textit{Discrete Mathematics and Its Applications}. McGraw-Hill, 2019.
\end{itemize}



\rule{6in}{.1pt}       % horizontal line 6 inches long and .1 point high

%---------------------------------------------------------------------

% \subsection makes a subsection heading; * leaves it unnumbered.
% (Usually subsections are inside sections, but the \section command
% used a font that was larger than I wanted.)
\subsection*{Exercises for Section 5.3:}     

\noindent
{\bf 6(a,c,d): (3 points). Do NOT do proofs.} For 6d it should read $n\geq 2$\\
\noindent
Determine whether each of these proposed definitions is a valid recursive definition of a function $f$ from the set of non-negative integers to the set of integers. If $f$ is well defined, find a formula for $f(n)$ when $n$ is a non-negative integer and prove that your formula is valid.

\noindent $f(0)=1,f(n)=-f(n-1)$ for $n\geq1$\\
Valid recursive definition.\\
$f(n)=(-1)^n$

\noindent $f(0)=0,f(1)=1,f(n)=2f(n+1)$ for $n\geq2$\\
Invalid recursive definition.

\noindent $f(0)=0,f(1)=1,f(n)=2f(n-1)$ for $n\geq2$\\
Valid recursive definition.\\
$f(2)=2f(1)=2,f(3)=2f(2)=4,\cdots$\\
$\displaystyle f(n)=2^{\left\lfloor \frac{n}{2}\right\rfloor}$

\noindent
{\bf 12: (2 points).}

\noindent
$f_n$ is the $n$th Fibonacci number.

\noindent
Prove that $\displaystyle f_1^2+f_2^2+\cdots+f_n^2=f_{n}f_{n+1}$ when $n$ is a positive integer.

\noindent
\textit{Base Case}:\\
$f_1^2=1^2=1, f_1\cdot f_2=1\cdot1=1$\\
True for $n=1$

\noindent
$f_1^2+f_2^2=1^2+1^2=2,f_2\cdot f_3=1\cdot2=2$\\
True for $n=2$

\noindent
\textit{Inductive Hypothesis}:\\
Assume $\displaystyle\sum_{i=1}^n f_i^2=f_n\cdot f_{n+1}$

\noindent
\textit{Inductive Step}:\\
$\displaystyle\sum_{i=1}^{n+1} f_i^2$\\
$\displaystyle=\sum_{i=1}^n f_i^2+f_{n+1}^2$\\
$\displaystyle=f_n\cdot f_{n+1}+f_{n+1}^2$\\
$=f_{n+1}(f_n+f_{n+1})$\\
$=f_{n+1}f_{n+2}$

\noindent
{\bf 18: (3 points).}\\

\noindent
Let\\
$A=\begin{bmatrix}
    1 & 1 \\
    1 & 0
\end{bmatrix}$

\noindent
Show that\\
$A^n=\begin{bmatrix}
    f_{n+1} & f_n \\
    f_n     & f_{n-1}
\end{bmatrix}$

\noindent
when $n$ is a positive integer.

\noindent
\textit{Base Case}:\\
$\displaystyle A^1=\begin{bmatrix}
    f_2 & f_1 \\
    f_1 & f_0
\end{bmatrix}=\begin{bmatrix}
    1 & 1 \\
    1 & 0
\end{bmatrix}$

\noindent
\textit{Inductive Hypothesis}:\\
Assume: $\displaystyle A^n=\begin{bmatrix}
    f_{n+1} & f_{n}   \\
    f_{n}   & f_{n-1} \\
\end{bmatrix}$ for all $n\geq2$

\noindent
\textit{Inductive Step}:\\
$A^{n+1}=A^n\times A^{1}=\begin{bmatrix}
    f_{n+1} & f_{n}   \\
    f_{n}   & f_{n-1} \\
\end{bmatrix}\times\begin{bmatrix}
    1 & 1 \\
    1 & 0
\end{bmatrix}=$\\
$\begin{bmatrix}
    f_{n+1}+f_{n} & f_n+f_{n-1} \\
    f_{n+1}       & f_n
\end{bmatrix}=\begin{bmatrix}
    f_{n+1} & f_{n+1} \\
    f_{n+1} & f_n
\end{bmatrix}$

\noindent
{\bf 46: (2 points).}\\
Use structural induction to show that $l(T)$, the number of leaves of a full binary tree $T$, is 1 more than $i(T)$, the number of internal vertices of $T$.

\noindent
The full binary tree consisting of a single vertex, denoted $\bullet$, clearly has 1 leaf and 0 internal vertices: $l(\bullet)=i(\bullet)+1$.

\noindent
Let $T_L$ be the full binary tree left of $\bullet$ and $T_R$ be the full binary tree right of $\bullet$.\\
Because $T_L$ and $T_R$ are connected with an internal vertex, we can define $i(T)$ as follows: $i(T)=i(T_L)+i(T_R)+1$.\\
Similarly, $l(T)$ can be defined as: $l(T)=l(T_L)+l(T_2)$. Combined with the base case, $l(T)=(i(T_L)+1)+(i(T_R)+1)$.\\
$l(T)=(i(T_L)+1)+(i(T_R)+1)$\\
$=i(T_L)+i(T_R)+1+1$\\
$=i(T)+1$

\noindent
With this, the total length of a full binary tree is 1 more than the number of internal vertices.

\subsection*{Exercises for Section 5.4:}     

In the following problems, assume that the following functions are defined:
\begin{itemize}
\item def isNull(T : Tree)  $\rightarrow$ bool:
\item def value(T : Tree)  $\rightarrow$ int:\# may not be called on Null tree
\item def left(T : Tree) $\rightarrow$ Tree: \# may not be called on Null tree
\item def right(T : Tree) $\rightarrow$ Tree: \# may not be called on Null tree
\end{itemize}
Also, assume a binary search tree. That is, given Tree $x$, all values in the left subtree are $\leq value(x)$ and all values in the right subtree are $> value(x)$.

\noindent
{\bf Custom 1: (3 points)}
Write a recursive method that finds the smallest item in the tree.
\begin{verbatim}
def findMin(T: Tree)
    if (isNull(T)):
        raise TypeError
    
    if isNull(left(T)):
        return value(T)
    
    return findMin(left(T))
\end{verbatim}

\noindent
{\bf Custom 2: (3 points)}
Write a recursive method that finds the sum of all the values in the tree.
\begin{verbatim}
def sum(T: Tree):
    if isNull(T):
        return 0
    
    return sum(left(T)) + value(T) + sum(right(T));
\end{verbatim}

\clearpage
\noindent
{\bf Custom 3: (2 points)}
Write a recursive method that returns $True$ if its integer parameter, $x$, is in the tree and $False$ otherwise.
\begin{verbatim}
def contains(x: value, T: Tree):
    if (isNull(T)):
        return False
    
    if value(T) == x:
        return True
    
    return contains(x, left(T)) or contains(x, right(T))
\end{verbatim}

\clearpage
\noindent
{\bf 50: (2 points)}

\noindent
Sort $3,5,7,8,1,9,2,4,6$ using the quick sort.
\begin{verbatim}
protected static int[] quicksort(int[] list) {
    if (list.length == 0)
        return [];
    if (list.length == 1)
        return list;
    
    int pivot = median(list);
    int[] left = [], right = [];
    for (int value : list) {
        if (value < pivot)
            left = left.append(value);
        if (value > pivot)
            right = right.append(value);
    }
    return quicksort(left).append(pivot).append(quicksort(right));
}
\end{verbatim}
\begin{enumerate}
    \item Pivot = 5
    \begin{itemize}
        \item Left = $\{3,1,2,4\}$
        \item Right = $\{7,8,9,6\}$
    \end{itemize}
    
    \item Quicksort Left
    \begin{enumerate}
        \item Pivot = 2
        \begin{itemize}
            \item Left = $\{1\}$
            \item Right = $\{3,4\}$
        \end{itemize}
        \item Quicksort Left\\
            List is length 1, do nothing
        \item Quicksort Right
        \begin{enumerate}
            \item Pivot = 3
            \begin{itemize}
                \item Left = $\{\}$
                \item Right = $\{4\}$
            \end{itemize}
            \item Quicksort Left\\
            List is empty do, nothing.
            \item Quicksort Right\\
            List is length 1, do nothing
            \item return Left, Pivot, Right\\
            $\{3,4\}$
        \end{enumerate}
        \item return Left, Pivot, Right\\
        $\{1,2,3,4\}$
    \end{enumerate}
    
    \item Quicksort Right
    \begin{enumerate}
        \item Pivot = 7
        \begin{itemize}
            \item Left = $\{6\}$
            \item Right = $\{8,9\}$
        \end{itemize}
        \item Quicksort Left\\
            List is length 1, do nothing
        \item Quicksort Right
        \begin{enumerate}
            \item Pivot = 8
            \begin{itemize}
                \item Left = $\{\}$
                \item Right = $\{9\}$
            \end{itemize}
            \item Quicksort Left\\
            List is empty do, nothing.
            \item Quicksort Right\\
            List is length 1, do nothing
            \item return Left, Pivot, Right\\
            $\{8,9\}$
        \end{enumerate}
        \item return Left, Pivot, Right\\
        $\{6,7,8,9\}$
    \end{enumerate}
    
    \item return Left, Pivot, Right\\
    $\{1,2,3,4,5,6,7,8,9\}$
\end{enumerate}


\end{document}
