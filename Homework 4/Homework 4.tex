% Comment lines start with %
% LaTeX commands start with \
% This template was provided by Jennifer Welch for CSCE 222-200, Honors, Spring 2015

\documentclass[12pt]{article}  % This is an article with font size 12-point

% Packages add features
\usepackage{times}     % font choice
\usepackage{amsmath}   % American Mathematical Association math formatting
\usepackage{amsthm}    % nice formatting of theorems
\usepackage{amssymb}    % provides some symbols
\usepackage{latexsym}  % provides some more symbols
\usepackage{fullpage}  % uses most of the page (1-inch margins)

\setlength{\parskip}{.1in}  % increase the space between paragraphs

\renewcommand{\baselinestretch}{1.1}  % increase the space between lines

% Convenient renaming of symbols for logic formulas
\newcommand{\NOT}{\neg}
\newcommand{\AND}{\wedge}
\newcommand{\OR}{\vee}
\newcommand{\XOR}{\oplus}
\newcommand{\IMPLIES}{\rightarrow}
\newcommand{\IFF}{\leftrightarrow}
\providecommand{\ceil}[1]{$\left \lceil #1 \right \rceil$}
\providecommand{\floor}[1]{$\left \lfloor #1 \right \rfloor$}
\newcommand{\powerset}[1]{\mathbb{P}(#1)}

% Actual content starts here.
\begin{document}

\begin{center}         % center all the material between begin and end
{\large                % use larger font
CSCE 222 (Carlisle), Discrete Structures for Computing \\  % \\ is line break
Spring 2022 \\
Homework 4}
\end{center}
\rule{6in}{.1pt}       % horizontal line 6 inches long and .1 point high
\begin{center}
{\large
Type your name below the pledge to sign\\
On my honor, as an Aggie, I have neither given nor received unauthorized aid on this academic work.\\
HUY QUANG LAI}
\end{center}

% blank line separates paragraphs.  First line of a paragraph is automatically
% indented.  

\rule{6in}{.1pt}       % horizontal line 6 inches long and .1 point high
                    
\noindent              % don't indent
{\bf Instructions:}    % \bf makes text boldface
                       % \em makes text emphasized (italics)

\begin{itemize}        % makes an itemized list
\item The exercises are from the textbook.  You are encouraged to work
      extra problems to aid in your learning; remember, the solutions to 
      the odd-numbered problems are in the back of the book.
\item Grading will be based on correctness, clarity, and whether your
      solution is of the appropriate length.
\item Always justify your answers.
\item Don't forget to acknowledge all sources of assistance in the section below, and write up your solutions on your own.
\item {\em Turn in .pdf file to Gradescope by the start of class on Monday, February 14, 2022.}  It is simpler to put each problem on its own page using the LaTeX clearpage command.
\end{itemize}


\rule{6in}{.1pt}       % horizontal line 6 inches long and .1 point high

{\bf Help Received:}    % \bf makes text boldface
\begin{itemize}
\item Rosen, Kenneth H. \emph{Discrete Mathematics and Its Applications}. McGraw-Hill, 2019. 
\end{itemize}



\rule{6in}{.1pt}       % horizontal line 6 inches long and .1 point high

%---------------------------------------------------------------------

% \subsection makes a subsection heading; * leaves it unnumbered.
% (Usually subsections are inside sections, but the \section command
% used a font that was larger than I wanted.)
%--------------------------------------------------------------------

\subsection*{Exercises for Section 2.3:}     

\noindent{\bf{8 (a,c,e,g): (2 points)}}\newline
\floor{1.1}$=1$\newline
\floor{-0.1}$=-1$\newline
\ceil{2.99}$=3$\newline
$\left\lfloor\frac{1}{2}+\left\lceil\frac{3}{2}\right\rceil\right\rfloor=2$

\noindent{\bf{20(a-d): (2 points)}}\newline
\noindent{Give an example of a function from $\mathbb{N}$ to $\mathbb{N}$ that is}
\begin{enumerate}
    \item one-to-one but not onto.\newline
        $f(x)=2x$
    \item onto but not one-to-one.\newline
        $\displaystyle f(x)=\begin{cases}
        1 & x=1\\
        x-1 & x\neq1
        \end{cases}
        $
    \item both onto and one-to-one (but different from the identity function).\newline
        $\displaystyle f(x)=\begin{cases}
        2 & x=1\\
        1 & x=2\\
        x & x>2
        \end{cases}
        $
    \item neither one-to-one nor onto.\newline
        $f(x)=1$
\end{enumerate}

\noindent{\bf{34(a): (2 points)}}\newline
\noindent{Suppose that $g$ is a function from $A$ to $B$ and $f$ is a function from $B$ to $C$. Prove each of these} statements.\newline
If $f\circ g$ is onto, then $f$ must also be onto.
Since $f\circ g$ is surjective, then for every $c\in C$ there exists $a\in A$ such that $f(g(a))=c$.\newline
Let $b=g(a)$.\newline
Then $b\in B$ and $f(b)=f(g(a))=c$.\newline
Thus $f(b)=c$. Hence $f$ is surjective.

\noindent{\bf{38: (2 points)}}\newline
\noindent{Find $f\circ g$ and $g\circ f$, where $f(x)=x^2+1$ and $g(x)=x+2$, are functions from $\mathbb{R}$ to $\mathbb{R}$.}\newline
$f\circ g=(x+2)^2+1=x^2+2x^2+5$\newline
$g\circ f=x^2+1+2=x^2+3$

\clearpage
\noindent{\bf{58: (2 points)}}\newline
\noindent{Let $a$ and $b$ be real numbers with $a<b$. Use the floor and/or ceiling functions to express the number of integers $n$ that satisfy the inequality $a\leq n\leq b$.}\newline
$a\leq n\IFF$\ceil{a}$\leq n$ and $n\leq b\IFF n\leq$\floor{b}\newline
Because of this, \ceil{a}$\leq n\leq$\floor{b}.\newline
The number of integers satisfying the inequality will be \floor{b}$-$\ceil{a}$+1$.

\subsection*{Exercises for Section 2.4:}     

\noindent{{\bf 4c: (1 point)}}\newline
\noindent{What are the terms $a_0,a_1,a_2,\textrm{and } a_3$ of the sequence $\{a_n\}$, where $a_n$ equals $7+4^n$.}\newline
$a_0=8,a_1=11,a_2=23,a_3=71$

\noindent{\bf{10d: (1 point)}}\newline
\noindent{Find the first six terms of the sequence defined by each of these recurrence relations and initial conditions.}\newline
$a_n=na_{n-1}+a_{n-2}^2,a_0=-1,a_1=0$\newline
$a_0=-1,a_1=0,a_2=1,a_3=3,a_4=13,a_5=74$

\noindent{\bf{14f: (1 points)}}\newline
\noindent{For each of these sequences find a recurrence relation satisfied by this sequence. (The answers are not unique because there are infinitely many different recurrence relations satisfied by any sequence.)}\newline
$a_n=n^2+n$\newline
$a_0=0$\newline
$a_n=a_{n-1}+2n$\newline

\noindent{\bf{18(a-c): (2 points)}}\newline
\noindent{A person deposits \$1000 in an account that yields 9\% interest compounded annually.}\newline
Set up a recurrence relation for the amount in the account at the end of n years.\newline
$a_0=1000$\newline
$a_n=1.09\cdot a_{n-1}$

\noindent{Find an explicit formula for the amount in the account at the end of n years.}\newline
$a_n=1000\cdot1.09^n$e

\noindent{How much money will the account contain after 100 years?}\newline
$a_{100}=\$5529040.79$

\clearpage
\noindent{\bf{22(a-c): (2 points)}}\newline
\noindent{An employee joined a company in 2017 with a starting salary of \$50,000. Every year this employee receives a raise of \$1000 plus 5\% of the salary of the previous year.}\newline
\noindent{Set up a recurrence relation for the salary of this employee n years after 2017.}\newline
$a_0=50000$\newline
$a_n=1.05a_{n+1}+1000$

\noindent{What will the salary of this employee be in 2025?}\newline
$2017, a_0=50000$\newline
$2018, a_1=50000\cdot1.05+1000=53500$\newline
$2019, a_2=53500\cdot1.05+1000=57175$\newline
$2020, a_3=57175\cdot1.05+1000=61033.75$\newline
$2021, a_4=61033.75\cdot1.05+1000=65065.4375$\newline
$2022, a_5=65065.4375\cdot1.05+1000=69318.709375$\newline
$2023, a_6=69318.709375\cdot1.05+1000=73784.6448437$\newline
$2024, a_7=73784.6448437\cdot1.05+1000=78473.8770859$\newline
$2025, a_8=78473.8770859\cdot1.05+1000=83397.5709402$\newline
\$$83397.57$

\noindent{Find an explicit formula for the salary of this employee n years after 2017.}\newline
$a_0=50000,k=1.05,d=1000$\newline
$a_1=ka_0+d$\newline
$a_2=ka_1+d=k(ka_0+d)+d=k^2a_0+(k+1)d$\newline
$a_3=ka_2+d=k(k^2a_0+(k+1)d)+d=k^3a_0+(k^2+k+1)d$\newline
From the coefficient for $d$, $\displaystyle\sum_{i=0}^{n}k^i=\frac{k^n-1}{k-1}$\newline
From the pattern, $\displaystyle a_n=a_0\cdot k^n+\frac{k^n-1}{k-1}d$\newline
Substituting values back into variables:\newline
\[
    a_n=50000\cdot1.05^n+\frac{1.05^n-1}{0.05}\cdot1000
\]

\clearpage
\noindent{\bf{24(a-b): (2 points)}}\newline
\noindent{Find a recurrence relation for the balance $B(k)$ owed at the end of $k$ months on a loan at a rate of $r$ if a payment $P$ is made on the loan each month. [Hint: Express $B(k)$ in terms of $B(k-1)$ and note that the monthly interest rate is $\frac{r}{12}$.]}\newline
$\displaystyle B(k)=\left(1+\frac{r}{12}\right)B(k-1)-P$\newline

\noindent{Determine what the monthly payment $P$ should be so that the loan is paid off after $T$ months.}\newline
Let $m=1+\frac{r}{12}$. Then,\newline
\begin{flalign*}
    B(k)&= mB(k-1)-P\\
        &= m(mB(k-2)-P)-P\\
        &= m^2B(k-2)-(m+1)P\\
        &= m^2(mB(k-3)-P)-(m+1)P\\
        &= m^3B(k-3)-(m^2+m+1)P\\
        &= m^kB(0)-\frac{m^k-1}{m-1}P\\
\end{flalign*}
Let $k=T$ such that $B(T)=0$\newline
$\displaystyle m^TB(0)-\frac{m^T-1}{m-1}P=0$\newline
$\displaystyle P=\frac{m^TB(0)(m-1)}{m^T-1}$\newline
Substituting for variables:
\[
    P=\frac{\left(\frac{r}{12}+1\right)^TB(0)\frac{r}{12}}{\left(\frac{r}{12}+1\right)^T-1}
\]

\noindent{\bf{40: (1 points)}}\newline
$\displaystyle\sum_{k=99}^{200}k^3$\newline
$\displaystyle=\sum_{k=1}^{200}k^3-\sum_{k=1}^{98}k^3$\newline
$\displaystyle=\frac{200^2(201)^2}{4}-\frac{98^2(99^2)}{4}$\newline
$\displaystyle=404010000-23532201$\newline
$=380477799$

\end{document}
