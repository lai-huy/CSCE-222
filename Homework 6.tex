% Comment lines start with %
% LaTeX commands start with \
% This template was provided by Jennifer Welch for CSCE 222-200, Honors, Spring 2015

\documentclass[12pt]{article}  % This is an article with font size 12-point

% Packages add features
\usepackage{times}     % font choice
\usepackage{amsmath}   % American Mathematical Association math formatting
\usepackage{amsthm}    % nice formatting of theorems
\usepackage{amssymb}    % provides some symbols
\usepackage{latexsym}  % provides some more symbols
\usepackage{fullpage}  % uses most of the page (1-inch margins)
\usepackage{multicol}
\setlength{\columnsep}{1cm}

\setlength{\parskip}{.1in}  % increase the space between paragraphs

\renewcommand{\baselinestretch}{1.1}  % increase the space between lines

% Convenient renaming of symbols for logic formulas
\newcommand{\NOT}{\neg}
\newcommand{\AND}{\wedge}
\newcommand{\OR}{\vee}
\newcommand{\XOR}{\oplus}
\newcommand{\IMPLIES}{\rightarrow}
\newcommand{\IFF}{\leftrightarrow}
\providecommand{\myceil}[1]{$\left \lceil #1 \right \rceil$}
\providecommand{\myfloor}[1]{$\left \lfloor #1 \right \rfloor$}
\newcommand{\powerset}[1]{\mathbb{P}(#1)}

% Actual content starts here.
\begin{document}

\begin{center}         % center all the material between begin and end
{\large                % use larger font
CSCE 222 (Carlisle), Discrete Structures for Computing \\  % \\ is line break
Spring 2022 \\
Homework 6}
\end{center}
\rule{6in}{.1pt}       % horizontal line 6 inches long and .1 point high
\begin{center}
{\large
Type your name below the pledge to sign\\
On my honor, as an Aggie, I have neither given nor received unauthorized aid on this academic work.\\
HUY QUANG LAI}
\end{center}

% blank line separates paragraphs.  First line of a paragraph is automatically
% indented.  

\rule{6in}{.1pt}       % horizontal line 6 inches long and .1 point high
                    
\noindent              % don't indent
{\bf Instructions:}    % \bf makes text boldface
                       % \em makes text emphasized (italics)

\begin{itemize}        % makes an itemized list
\item The exercises are from the textbook.  You are encouraged to work
      extra problems to aid in your learning; remember, the solutions to 
      the odd-numbered problems are in the back of the book.
\item Grading will be based on correctness, clarity, and whether your
      solution is of the appropriate length.
\item Always justify your answers.
\item Don't forget to acknowledge all sources of assistance in the section below, and write up your solutions on your own.
\item {\em Turn in .pdf file to Gradescope by the start of class on Monday, February 28, 2022.}  It is simpler to put each problem on its own page using the LaTeX clearpage command.
\end{itemize}


\rule{6in}{.1pt}       % horizontal line 6 inches long and .1 point high

{\bf Help Received:}    % \bf makes text boldface
\begin{itemize}
\item Rosen, Kenneth H. \emph{Discrete Mathematics and Its Applications}. McGraw-Hill, 2019.
\end{itemize}



\rule{6in}{.1pt}       % horizontal line 6 inches long and .1 point high

%---------------------------------------------------------------------

% \subsection makes a subsection heading; * leaves it unnumbered.
% (Usually subsections are inside sections, but the \section command
% used a font that was larger than I wanted.)


%--------------------------------------------------------------------

\subsection*{Exercises for Section 3.2:}     

\noindent{\bf{6: (2 points)}}\\
\noindent{Show that $\displaystyle\frac{x^3+2x}{2x+1}$ is $\displaystyle O(x^2)$}\\
To satisfy the definition of $O(x^2)$, we need to find an appropriate choice of $c$ and $k$.\\
Let’s consider $x\geq1000$ and reason by inequalities:
\[\frac{x^3+2x}{2x+1}<\frac{x^3+2x}{2x}<\frac{1}{2}x^2+x^2=\frac{3}{2}x^2\]
Since $\displaystyle\frac{x^3+2x}{2x+1}<\frac{3}{2}x^2$ when $x\geq1000$, we can choose $k=1000$ and $c=\frac{3}{2}$

\noindent{\bf{22: (2 points)}}\\
\noindent{Arrange the functions $(1.5)^n,n^{100},\log^3{n},\sqrt{n}\log{n},10^n,(n!)^2,\textrm{ and } n^{99} + n^{98}$ in a list so that each function is big-$O$ of the next function.}\\
$\log^3{n},\sqrt{n}\log{n},n^{99}+n^{98}, n^{100},(1.5)^n,10^n,(n!)^2$

\noindent{\bf{26(a-c): (2 points)}}\\
\noindent{Give a big-$O$ estimate for each of these functions. For the function $g$ in your estimate $f(x)$ is $O(g(x))$, use a simple function $g$ of smallest order.}
\begin{enumerate}
    \item $(n^3+n^2\log{n})(\log{n}+1)+(17\log{n}+19)(n^3+2)$\\
    $O(n^3\log{n})$
    \item $(2^n+n^2)(n^3+3^n)$\\
    $O(6^n)$
    \item $(n^n+n2^n+5^n)(n!+5^n)$\\
    $O(n^n\cdot n!)$
\end{enumerate}

\noindent{\bf{74: (2 points)}}\\
\noindent{Determine whether $\log{n!}$ is $\Theta(n\log{n})$. Justify your answer.}\\
$\displaystyle\log{n!}=\log{1}+\log{2}+\log{3}+\cdots+\log{n}$\\
\\
Upper Bound:\\
$\displaystyle\log{1}+\log{2}+\log{3}+\cdots+\log{n}\leq\log{n}+\log{n}+\log{n}+\cdots\log{n}$\\
$\displaystyle=n\cdot\log{n}$\\
\\
Lower Bound:\\
$\displaystyle\log{1}+\cdots+\log{\frac{n}{2}}+\cdots+\log{n}\geq\log{\frac{n}{2}}+\log{\left(\frac{n}{2}+1\right)}+\cdots+\log{\left(n-1\right)}+\log{n}$\\
$\displaystyle=\frac{n}{2}\cdot\log{\left(\frac{n}{2}\right)}$

\subsection*{Exercises for Section 3.3:}     

\noindent{\bf{2: (1 point).}}\\
\noindent{Give a big-$O$ estimate for the number additions used in this segment of an algorithm.}
\begin{verbatim}
t:=0
for i:=1 to n
    for j:=1 to n
        t:=t+i+j 
\end{verbatim}
$O(n^2)$

\noindent{\bf{4: (1 point).}}\\
\noindent{Give a big-$O$ estimate for the number of operations, where an operation is an addition or a multiplication, used in this segment of an algorithm (ignoring comparisons used to test the conditions in the while loop).}
\begin{verbatim}
i := 1
t := 0
while i <= n
    t := t + 1
    i := 2i
\end{verbatim}
Addition: $O(\log{n})$\\
Multiplication: $O(\log{n})$\\

\noindent{\bf{8: (2 points).}}\\
\noindent{Given a real number $x$ and a positive integer $k$, determine the number of multiplications used to find $x^{2^k}$ starting with $x$ and successively squaring (to find $x^2,x^4$, and so on). Is this a more efficient way to find $x^{2^k}$ than by multiplying $x$ by itself the appropriate number of times?}\\
Successively squaring will double the exponent of $x^2$ $k$ number of times. Because of this, successively squaring would be $O(k)$.\\
This is more efficient than multiplying $x$ $2^k$ number of times as the number of multiplication which is $O(2^k)$

\clearpage
\noindent{\bf{12b: (2 points).}}\\
\noindent{Consider the following algorithm, which takes as input a sequence of $n$ integers $a_1, a_2,\cdots,a_n$ and produces as output a matrix $M=\{m_{ij}\}$ where $m_{ij}$ is the minimum term
in the sequence of integers $a_i, a_{i+1},\cdots,a_j$ for $j\geq i$ and $m_{ij}=0$ otherwise.}
\begin{tabbing}
initialize $M$ so that $m_{ij}=a_i$ if $j\geq i$ and $m_{ij}=0$ otherwise\\
for \=$i$ := 1 to $n$.\\
\> for \= $j$ := i + 1 to $n$.\\
\> \>for \= $k$ := i + 1 to $j$\\
\> \> \>$m_{ij}$ := min($m_{ij},a_k$)\\
return $M={m_{ij}}$ \{$m_{ij}$ is the minimum term of $a_i,a_{i+1},\cdots,a_j$\}
\end{tabbing}
Show that this algorithm uses $\Omega(n^3)$ comparisons to compute the matrix $M$. Using this fact and part (a), conclude that the algorithms uses $\Theta(n^3)$ comparisons. [Hint: Only consider the cases where $i\leq \frac{n}{4}$ and $j\geq \frac{3n}{4}$ in the two outer loops in the algorithm.]\\

\noindent{The outermost loop ($i$) will run more than $\frac{n}{4}$ times.}
In other words, $i\geq\frac{n}{4}$ Because of this, the second nested loop ($j$) will run at least $1-i$ times.
In other words, $j\geq\frac{3n}{4}$. A similar argument can be applied to the innermost loop $k$ to get $k\geq\frac{3n}{4}$.
Because of this, the algorithm will loop more than $i\times j\times k$ times.\\
$\displaystyle f(x)\geq \frac{n}{4}\cdot\frac{3n}{4}\cdot\frac{3n}{4}$\\
$\displaystyle f(x)\geq \frac{9}{64}n^3$\\
Because of this, the algorithm is $\Omega(n^3)$.\\
Since the algorithm is both $O(n^3)\AND\Omega(n^3)$, the algorithm must also be $\Theta(n^3)$

\noindent{\bf{14a: (1 points)}}\\
\noindent{There is a more efficient algorithm (in terms of the number of multiplications and additions used) for evaluating polynomials than the conventional algorithm described in the previous exercise. It is called \textbf{Horner’s method}. This pseudocode shows how to use this method to find the value of $a_nx^n+a_{n-1}x^{n-1}+\cdots+a_1x+a_0$ at $x=c$.}
\begin{tabbing}
\textbf{procedure} Horner($c,a_0,a_1,a_2,\cdots,a_n$: real numbers)\\
$y:=a_n$\\
for \=$i:=1 \textrm{ to } n$\\
\>$y:=y\cdot c+a_{n-i}$\\
return $y\{y=a_nc^n+a_{n-1}c^{n-1}+\cdots+a_1c+a_0\}$
\end{tabbing}
Evaluate $3x^2+x+1$ at $x=2$ by working through each step of the algorithm showing the values assigned at each assignment step.
\clearpage
\begin{flalign*}
    y&:=2               & \textrm{Initial}&\textrm{ Condition}  &&\\
    y&:=2\cdot2+1=7     & i&=1                                  &&\\
    y&:=7\cdot2+1=15    & i&=2 \textrm{ return 15}              &&
\end{flalign*}
$3x^2+x+1$ evaluated at $x=2$ is $15$.

\noindent{\bf{14b: (1 points)}}\\
\noindent{Exactly how many multiplications and additions are used by this algorithm to evaluate a polynomial of degree $n$ at $x=c$? (Do not count additions used to increment the loop variable.)}\\
$n$ multiplications and $n$ additions.\\
One each in step which we do $n$ times in the for-loop.

\noindent{\bf{20(b,c,e,g): (2 points)}}
\begin{multicols}{2}
[
\noindent{What is the effect in the time required to solve a problem when you double the size of the input from $n$ to $2n$, assuming that the number of milliseconds the algorithm uses to solve the problem with input size $n$ is each of these functions? [Express your answer in the simplest form possible, either as a ratio or a difference. Your answer may be a function of $n$ or a constant.]}
]

\noindent{$\log{n}$}
\begin{flalign*}
     & \log{2n}-\log{n}\\
    =& \log{\frac{2n}{n}}\\
    =& \log{2}&&
\end{flalign*}

\noindent{$100n$}
\begin{flalign*}
     & \frac{100(2n)}{100n}\\
    =& 2&&
\end{flalign*}

\noindent{$n^2$}
\begin{flalign*}
     & \frac{(2n)^2}{n^2}\\
    =& 4&&
\end{flalign*}

\noindent{$2^n$}
\begin{flalign*}
     & \frac{2^{2n}}{2^n}\\
    =& 2^n&&
\end{flalign*}
\end{multicols}

\clearpage
\noindent{\bf{42: (2 points)}}\\
\noindent{Find the complexity of the greedy algorithm for scheduling the most talks by adding at each step the talk with the earliest end time compatible with those already scheduled (Algorithm 7 in Section 3.1). Assume that the talks are not already sorted by earliest end time and assume that the worst-case time complexity of sorting is $O(n\log{n})$.}\\

\noindent{The first step of the greedy algorithm would be to sort the lest by end-time. This would take, at most, $O(n\log{n})$.}\\
Then the greedy algorithm would traverse the list backwards to greedily schedule meetings by end-time. This process would take, at most, $O(n)$.\\
Therefore, the algorithm as a whole would take, at most, $O(n\log{n}+n)$ which can be simplified to $O(n\log{n})$.

\end{document}
