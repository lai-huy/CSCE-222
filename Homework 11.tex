% Comment lines start with %
% LaTeX commands start with \
% This template was provided by Jennifer Welch for CSCE 222-200, Honors, Spring 2015

\documentclass[12pt]{article}  % This is an article with font size 12-point

% Packages add features
\usepackage{times}     % font choice
\usepackage{amsmath}   % American Mathematical Association math formatting
\usepackage{amsthm}    % nice formatting of theorems
\usepackage{amssymb}    % provides some symbols
\usepackage{latexsym}  % provides some more symbols
\usepackage{fullpage}  % uses most of the page (1-inch margins)
\usepackage[shortlabels]{enumitem}

\setlength{\parskip}{.1in}  % increase the space between paragraphs

\renewcommand{\baselinestretch}{1.1}  % increase the space between lines

% Convenient renaming of symbols for logic formulas
\newcommand{\NOT}{\neg}
\newcommand{\AND}{\wedge}
\newcommand{\OR}{\vee}
\newcommand{\XOR}{\oplus}
\newcommand{\IMPLIES}{\rightarrow}
\newcommand{\IFF}{\leftrightarrow}
\providecommand{\myceil}[1]{$\left \lceil #1 \right \rceil$}
\providecommand{\myfloor}[1]{$\left \lfloor #1 \right \rfloor$}
\newcommand{\powerset}[1]{\mathbb{P}(#1)}

% Actual content starts here.
\begin{document}

\begin{center}         % center all the material between begin and end
{\large                % use larger font
CSCE 222 (Carlisle), Discrete Structures for Computing \\  % \\ is line break
Spring 2022 \\
Homework 11}
\end{center}
\rule{6in}{.1pt}       % horizontal line 6 inches long and .1 point high
\begin{center}
{\large
Type your name below the pledge to sign\\
On my honor, as an Aggie, I have neither given nor received unauthorized aid on this academic work.\\
**HUY QUANG LAI**}
\end{center}

% blank line separates paragraphs.  First line of a paragraph is automatically
% indented.  

\rule{6in}{.1pt}       % horizontal line 6 inches long and .1 point high
                    
\noindent              % don't indent
{\bf Instructions:}    % \bf makes text boldface
                       % \em makes text emphasized (italics)

\begin{itemize}        % makes an itemized list
\item The exercises are from the textbook.  You are encouraged to work
      extra problems to aid in your learning; remember, the solutions to 
      the odd-numbered problems are in the back of the book.
\item Grading will be based on correctness, clarity, and whether your
      solution is of the appropriate length.
\item Always justify your answers.
\item Don't forget to acknowledge all sources of assistance in the section below, and write up your solutions on your own.
\item {\em Turn in .pdf file to Gradescope by the start of class on Monday, April 18, 2022.}  It is simpler to put each problem on its own page using the LaTeX clearpage command.
\end{itemize}


\rule{6in}{.1pt}       % horizontal line 6 inches long and .1 point high

{\bf Help Received:}    % \bf makes text boldface
\begin{itemize}
\item Rosen, Kenneth H. \textit{Discrete Mathematics and Its Applications}. McGraw-Hill, 2019.
\end{itemize}



\rule{6in}{.1pt}       % horizontal line 6 inches long and .1 point high

%---------------------------------------------------------------------

% \subsection makes a subsection heading; * leaves it unnumbered.
% (Usually subsections are inside sections, but the \section command
% used a font that was larger than I wanted.)
\subsection*{Exercises for Section 4.1:}     

\noindent{\bf 38(a-d): (2 points).}

\noindent Find each of these values.
\begin{enumerate}[a)]
    \item $(19^2\bmod{41})\bmod{9}$\\
    $=2$
    \item $(32^3\bmod{13})^2\bmod{11}$\\
    $=9$
    \item $(7^3\bmod{23})^2\bmod{31}$\\
    $=7$
    \item $(21^2\bmod{15})^2\bmod{22}$\\
    $=14$
\end{enumerate}

\subsection*{Exercises for Section 4.2:}     

\noindent{\bf Express the octal number 1437 in binary, decimal and hexadecimal: (1 point).}

\noindent
$1437_8=001100011111_{2}$\\
$1437_8=31F_{16}$\\
$1437_8=799_{10}$

\noindent{\bf 26: (2 points).}

\noindent
Use Algorithm 5 to find $11^{644}\bmod{645}$

\noindent
$i=0$: Because $a_0=0$, we have $x=1$ and $\textrm{\textit{power}}=11^2\bmod{645}=11\bmod{645}=121$\\
$i=1$: Because $a_1=0$, we have $x=1$ and $\textrm{\textit{power}}=121^2\bmod{645}=14641\bmod{645}=451$\\
$i=2$: Because $a_2=1$, we have $x=1\cdot451\bmod{645}=451$ and $\textrm{\textit{power}}=451^2\bmod{645}=226$\\
$i=3$: Because $a_3=0$, we have $x=451$ and $\textrm{\textit{power}}=226^2\bmod{645}=121$\\
$i=4$: Because $a_4=0$, we have $x=451$ and $\textrm{\textit{power}}=121^2\bmod{645}=451$\\
$i=5$: Because $a_5=0$, we have $x=451$ and $\textrm{\textit{power}}=451^2\bmod{645}=226$\\
$i=6$: Because $a_6=0$, we have $x=451$ and $\textrm{\textit{power}}=226^2\bmod{645}=121$\\
$i=7$: Because $a_7=1$, we have $x=451\cdot121\bmod{645}=391$ and $\textrm{\textit{power}}=121^2\bmod{645}=451$\\
$i=8$: Because $a_8=0$, we have $x=391$ and $\textrm{\textit{power}}=451^2\bmod{645}=226$\\
$i=9$: Because $a_9=1$, we have $x=391\cdot226\bmod{645}=1$

\clearpage
\subsection*{Exercises for Section 4.3:}     

\noindent
{\bf 24(a-b): (1 point).}

\noindent
What are the greatest common divisors of these pairs of integers?
\begin{enumerate}[a)]
    \item $2^2\cdot3^3\cdot5^5,2^5\cdot3^3\cdot5^2$\\
    $2^2\cdot3^3\cdot5^2$
    \item $2\cdot3\cdot5\cdot7\cdot11\cdot13,2^{11}\cdot3^9\cdot11\cdot17^{14}$\\
    $2\cdot3\cdot11$
\end{enumerate}


\noindent
{\bf 32(d-e): (2 points).}

\noindent
Use the Euclidean algorithm to find
\begin{enumerate}[a)]\setcounter{enumi}{3}
    \item $\gcd(1529,14039)$
    \begin{flalign*}
        \textrm{a:} & 1529  & \textrm{b:} & 14039   &\\
        \textrm{a:} & 14039 & \textrm{b:} & 1529    &\\
        \textrm{a:} & 1529  & \textrm{b:} & 278     &\\
        \textrm{a:} & 278   & \textrm{b:} & 139     &\\
        \textrm{a:} & 139   & \textrm{b:} & 0       &
    \end{flalign*}
    $\gcd=139$
    
    \item $\gcd(1529,14038)$
    \begin{flalign*}
        \textrm{a:} & 1529  & \textrm{b:} & 14038   & \\
        \textrm{a:} & 14038 & \textrm{b:} & 1529    & \\
        \textrm{a:} & 1529  & \textrm{b:} & 277     & \\
        \textrm{a:} & 277   & \textrm{b:} & 144     & \\
        \textrm{a:} & 144   & \textrm{b:} & 133     & \\
        \textrm{a:} & 133   & \textrm{b:} & 11      & \\
        \textrm{a:} & 11    & \textrm{b:} & 1       & \\
        \textrm{a:} & 1     & \textrm{b:} & 0
    \end{flalign*}
    $\gcd=1$
    
\end{enumerate}

\clearpage
\noindent{\bf 40(d-e): (2 points).}

\noindent
Using the method followed in Example 17, express the greatest common divisor of each of these pairs of integers as a linear combination of these integers.
\begin{enumerate}[a)]\setcounter{enumi}{3}
    \item $21,55$\\
    $55=21\cdot2+13$\\
    $21=13\cdot1+8$\\
    $13=8\cdot1+5$\\
    $8=5\cdot1+3$\\
    $5=3\cdot1+2$\\
    $3=2\cdot1+1$\\
    $2=1\cdot2+0$
    
    \noindent
    $1=3-1\cdot2$\\
    $1=3-1\cdot(5-3)$\\
    $1=2\cdot3-1\cdot5$\\
    $1=2\cdot(8-1\cdot5)-1\cdot5$\\
    $1=2\cdot8-3\cdot5$\\
    $1=2\cdot8-3\cdot(13-1\cdot8)$\\
    $1=5\cdot(21-1\cdot13)-3\cdot13$\\
    $1=5\cdot21-8\cdot13$\\
    $1=5\cdot21-8(55-2\cdot21)$\\
    $1=21\cdot21-8\cdot55$
    
    \noindent
    $\gcd{(21,55)}=1=21\cdot21-8\cdot55$
    
    \item $101,203$\\
    $203=101\cdot2+1$\\
    $101=1\cdot101+0$
    
    \noindent
    $\gcd{(101,203)}=1=203-2\cdot101$
    
\end{enumerate}

\clearpage
\subsection*{Exercises for Section 4.4:}     

\noindent
{\bf 6(a,c): (1 point).}

\noindent
Find an inverse of $a$ modulo $m$ for each of these pairs of relatively prime integers using the method followed in Example 2.
\begin{enumerate}[a)]
    \item $a=2,m=17$
    \begin{flalign*}
    17 &= 8\cdot2+1 & \\
    2 &= 2\cdot1+0  &
    \end{flalign*}
    $1=17-8\cdot2$\\
    $-8\cdot2\bmod{17}=1$\\
    $9\cdot2\bmod{17}=1$\\
    $\bar{a}=9$
    
    \addtocounter{enumi}{1}
    \item $a=144,m=233$
    \begin{flalign*}
    233 &= 1\cdot144+89 & \\
    144 &= 1\cdot89+55  & \\
    89  &= 1\cdot55+34  & \\
    55  &= 1\cdot34+21  & \\
    34  &= 1\cdot21+13  & \\
    21  &= 1\cdot13+8   & \\
    13  &= 1\cdot8+5    & \\
    8   &= 1\cdot5+3    & \\
    5   &= 1\cdot3+2    & \\
    3   &= 1\cdot2+1    &
    \end{flalign*}
\end{enumerate}

\noindent
{\bf 20: (2 points).}

\noindent
Use the construction in the proof of the Chinese remainder theorem to find all solutions to the system of concurrences $x\equiv2(\bmod{3}),x\equiv1(\bmod{4})$, and $x\equiv3(\bmod{5})$.

\noindent
$\textrm{lcm}(3,4,5)=60$\\
$x=2\cdot20\cdot20^{-1}\bmod{3}+1\cdot15\cdot15^{-1}\bmod{4}+3\cdot12\cdot12^{-1}\bmod{5}$\\
$x=2\cdot20\cdot2+1\cdot15\cdot3+3\cdot12\cdot3$\\
$x=233\bmod{60}$\\
$x=53$

\clearpage
\subsection*{Exercises for Section 4.5:}     

\noindent
{\bf 4: (1 point).}

\noindent
Use the double hashing procedure we have described with $p=4969$ to assign memory locations to files for employees with social security numbers  $k_1=132489971,k_2=509496993,\\k_3=546332190,k_4=034367980,k_5=047900151,k_6=329938157,k_7=212228844,\\k_8=325510778,k_9=353354519,k_{10}=053708912$.

\noindent
$h(k)=k\bmod{p}$\\
$h(k_1)=132489971\bmod{4969}$\\
$h(k_1)=1524$

\noindent
$h(k_2)=509496993\bmod{4969}$\\
$h(k_2)=578$

\noindent
$h(k_3)=546332190\bmod{4969}=578$\\
$g(k_3)=546332191\bmod{4967}=1927$\\
$h(k,i)=(578+1(1927))\bmod{4969}=2505$

\noindent
$h(k_4)=034367980\bmod{4969}=2376$

\noindent
$k_5=3960$\\
$k_6=1526$\\
$k_7=2854$\\
$k_8=4927$\\
$k_9=1131$\\
$k_{10}=4702$

\noindent
{\bf 20(a-d): (2 points).}

\noindent
One digit in each of these identification numbers of a postal money order is smudged. Can you recover the smudged digit, indicated by a $Q$, in each of these numbers?
\begin{enumerate}[a)]
    \item $Q1223139784$\\
    $4=(Q+1+2+2+3+1+3+9+7+8)\bmod{9}$\\
    $4=(Q+36)\bmod{9}$\\
    $4=Q\bmod{9}$\\
    $Q=4$
    
    \clearpage
    \item $6702120Q988$\\
    $8=(6+7+0+2+1+2+0+Q+9+8)\bmod{9}$\\
    $8=(Q+35)\bmod{9}$\\
    $8=Q\bmod{9}+8$\\
    $0=Q\bmod{9}$\\
    $Q=\{0,9\}$
    
    \item $27Q41007734$
    $4=(2+7+Q+4+1+0+0+7+7+3)\bmod{9}$\\
    $4=(Q+31)\bmod{9}$\\
    $4=Q\bmod{9}+4$\\
    $Q=\{0,9\}$
    
    \item $213279032Q1$\\
    $1=(2+1+3+2+7+9+0+3+2+Q)\bmod{9}$\\
    $1=(Q+29)\bmod{9}$\\
    $-1=Q\bmod{9}$\\
    $Q=8$
\end{enumerate}

\clearpage
\subsection*{Exercises for Section 4.6:}     

\noindent
{\bf 8: (1 point).}

\noindent
Suppose that the ciphertext DVE CFMV KF NFEUVI, REU KYRK ZJ KYV JVVU FW JTZVETV was produced by encrypting a plaintext message using a shift cipher. What is the original plaintext?

\noindent
Shift 9 left or 17 right.\\
MEN LOVE TO WONDER, AND THAT IS THE SEED OF SCIENCE

\noindent
{\bf 18: (1 point).}

\noindent
Use the Vigen\`ere cipher with key BLUE to encrypt the message SNOWFALL.

\noindent
Key = BLUEBLUE\\
TYIACLFP

\noindent
{\bf 26: (2 points).}

\noindent
What is the original message encrypted using the RSA system with $n=53\cdot61$ and $e=17$ if the encrypted message is 3185 2038 2460 2550? (To decrypt, first find the decryption exponent $d$, which is the inverse of $e=17\bmod{52\cdot60}$.)

\noindent
$d=2753$\\
$53\cdot61=3233$\\
$3185^{d}\bmod{3233}=1816$\\
$2038^{d}\bmod{3233}=2008$\\
$2460^{d}\bmod{3233}=1717$\\
$2550^{d}\bmod{3233}=0411$

\noindent
$18,16,20,08,17,17,04,11$\\
SQUIRREL

\end{document}
